\documentclass{article}
\usepackage[margin=1in]{geometry}
\usepackage{listings}
\usepackage{hyperref}
\title{dOvs - Codegeneration}

\author{
  Group 9 \\
  Miran Hasanagi\'{c} - 20084902 \\
  Jakob Graugaard Laursen - 20093220\\
  Steven Astrup S\o rensen - 201206081
}

\begin{document}
\maketitle

\section{Introduction}
This report describes the work carried out in order to develop the code generation phase, which is the last component in the compiler for this course. The code generation component takes the Intermediate Representation (IR), and creates assembly code. However, it should be noted that the IR generated from the previous module, is used as input for the \texttt{canon.sml} module, in order to produce a simplified IR, otherwise known as basic blocks, according the chapter 8 in the book.

The rest of this report is structured the following way. The two following sections describes the two main sml files which have been developed, separately. Afterwards the experience gain and different tests are presented. Finally, the implementation of the code generation component is concluded.


\section{Instruction selection}
%How did you implement the missing elements in x86gen. Describe how the file works as a whole and explain a few interesting cases.

The goal of the code generation is to generated assembly code as specified in the assignment for this course. Hence the target assembly is x86. The \texttt{x86gen.sml} takes the output from the canon module, and creates the assembly code. Its main structure is similar to that of \texttt{semant.sml} and \texttt{irgen.sml}, in that recursive functions are used to transverse the tree. However, this module does not use ``real'' registers, but just assumes there are an unlimited amount of them. It relies on the support module \texttt{x86frame.sml} to create this illusion.

\section{The support modules}
%Write how you implemented the missing elements in x86frame..
As mentioned above, the \texttt{x86frame.sml} is a support module for the \texttt{x86gen.sml} module. The main responsibility of this module is to create the illusion of an infinite amount of registers for \texttt{x86gen.sml}. Each time \texttt{x86gen.sml} uses a ``register'', \texttt{x86frame} checks if is a real one. Those who isn't real is called temporaries. If a temporary was selected, \texttt{x86frame} looks for an available real one, and substitutes that register in. If the temporary was needed because it held a value, \texttt{x86frame.sml} loads the value from memory into the selected register, then allows the code to run. If the temporary instead was used to store a value, the selected code is allowed to run, then \texttt{x86frame.sml} stores the value back in memory from the selected register. 

This module both has implemented functions to check if it is a register, and if a specific register is unused currently, meaning that it selects a real register unused by the code.

\section{Exponentiation}
We were required to implement the runtime function of exponentiation. There are two cases to consider: Negative exponent and 0 as exponent with 0 as base.

The latter case is simple: It is always the case that a 0 exponent gives 1, and we do adopt this convension for $0^0$ as well.

The other case is a bit harder. However, since negative exponent means the inverse to some integer, in proper math this would be a value with absolute value $ < 1$. Reading around on wikipedia (and listening to our instructor), we get the \href{https://en.wikipedia.org/wiki/IEEE_floating_point}{recommendation} to the nearest even number. Hence, we return 0.

Should our base be $1$ or $-1$, we just go ahead and treat the exponent as its absolute value. This will still give us the proper value from am mathematical point of view.

The code is simply
\begin{verbatim}
int exponent (int base, int expn) {

    if (expn < 0 && (base != 1 || base != -1) {
        return 0;
    }

    int res = 1;

    while (expn != 0) {
        /* Exploit binary representation of expn */
        res *= res;
        if (expn % 2 != 0) {
            res *= base;
        }
        expn /= 2;
    }
    return res;
}
\end{verbatim}
\section{Problems encountered \& experience gained}
%If you had any problems in the work process, write it here with what you gained.

\subsection{General Experience}
It was hard to understand how to apply the canon module in the beginning, and also to generate the assembly code. However, the main module provided valuable help for both cases, using the compile method. Afterwards, the simple constructs were made, such a integer expression and binary operators, and tested. In general the approach was to create a tiger program, and then replace the relevant TODO inside both files, described above.
Next the more challenging constructs could be implemented. Some of the more challenging are addressed below.  

\subsection{TODO: Add more if needed}

\section{5 tiger programs}
%5 interesting tiger programs that you find test your codegeneration in a good way.

\subsection{test01.tig}

\lstinputlisting[frame=single]{tests/test01.tig}

\subsection{test02.tig}

\lstinputlisting[frame=single]{tests/test02.tig}

\subsection{test03.tig}

\lstinputlisting[frame=single]{tests/test03.tig}

\subsection{test04.tig}

\lstinputlisting[frame=single]{tests/test04.tig}

\subsection{test05.tig}

\lstinputlisting[frame=single]{tests/test05.tig}

\section{Conclusion}

This report presented the development of the code generator in \texttt{x86gen.sml}, which generates assembly instructions from the previous module. The main functionality was described, and how some modules, like \texttt{x86frame.sml}, are used in order to use real registers when generating assembly code. Additionally the gained experience was described. Finally, additional tests provided confidence that this code generator is working correctly.

\end{document}