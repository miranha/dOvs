\documentclass[a4paper]{article}
\usepackage[margin=1in]{geometry}
\title{dOvs - warmup project}
\author{
  Group 9 \\
  Miran Hasanagi\'{c} - 20084902 \\
  Jakob Graugaard Laursen - 20093220\\
  Steven Astrup S\o rensen - 201206081
}

\begin{document}
\maketitle

\section{Introduction}

This report describes the approach in order to develop a straight line interpreter. Additionally, it provides an overview of problems encountered and intersting experiments during development process. Finally, five test expressions are interpreted and shown.


%\section{How did we make the interpreter}
\section{Development process and collaboration}
\fbox{You should not write how you wrote the interpreter, but why you wrote it that way}
The group members met physically in order to both unterstand the problem and furthermore outline and discuss the solutions.

During the development different parts raised different challenges. \fbox{Describe the challenges here}. After the main functionality of the interpreter was woking correctly, different experiments led to the improvement of different special cases. For example, the two exceptions \texttt{DivisionByZero} and \texttt{IdExpNotFound}. Both can be interpreted with valid syntax, but can not be interpreted to valid values. Hence they had to be handled. The syntax is ensured by the \textbf{datatype}s \texttt{stm} and \texttt{exp}. Hence if a \texttt{stm} uses non valid constructs the SML interpreter will give an error.

Generally the suggestions in the book were followed. Since a \texttt{stm} can be viewed as having a TREE structure as described in the book, it was naturally to use mutual recursive functions in order to interpret an expression, such as \texttt{prog} from the book. In the end different parts of the code were refactored in order to make them more readable and use more fitting SML syntax. As an example \texttt{if} expressions were change to use pattern matching in relevant places. 

\section{Problems experienced}
\fbox{If you encountered any problems describe how you solved them}

This is already mentioned challenges in the previous section, maybe we should move it to here?? What do you guys think??

\section{Additional statements tested}
\fbox{5 statements as source code, and as values of type 'stm' plus a description of test including the outcome.}

\section{Conclusion}
This report described the development process of the straight line interpreter. This project was used as a ``warum up'' project for both learning programming in SML and general principles of interpreting a language. Both aspects provide valuable lessons in order to start work on the compiler project. 

\end{document}
