\documentclass{article}
\usepackage[margin=1in]{geometry}
\usepackage{listings}
\title{dOvs - Parsing}

\author{
  Group 9 \\
  Miran Hasanagi\'{c} - 20084902 \\
  Jakob Graugaard Laursen - 20093220\\
  Steven Astrup S\o rensen - 201206081
}

\begin{document}
\maketitle

\section{Introduction}
This report describes our approach to develop the parser, which is the first component in the compilation process. This parser is implemented using the ML-Yacc, as it is suggested by the book and the project description.

The first three section describe different parts of the parser development.
First, important parts of the tiger grammar are introduced with respect to the ML-Yacc. Secondly, it is described how the semantics actions are used to construct the AST for the program. Then it is shown how conflicts where detected, and what was done in order to resolve them.

Following these sections an overview of the problems encountered and experience gain during the work process is provided. Finally, five tiger programs are provided as test cases for different aspects of the developed parser.

\section{The tiger grammar}
%You should not write how you extracted the grammar from the appendix.

The tiger grammar was implemented in the file \texttt{tiger.grm}. The parser development was divided into two separate parts; first the tiger grammars was implemented, and next the semantics action where added. These two parts reflect the two \textit{program} section for the chapters three and four, respectively. Using this approach it was possible to divide the problem in two parts. First it could be ensured that we did not have any conflicts in the grammars, and afterwards the semantics actions could be implemented in order to create the Abstract Syntax Tree (AST). 

In order to specify the grammar and rules Appendix A was used. However, using this grammar directly provided some conflicts and problem, which is discussed below.  

\section{The abstract syntax}
%Write how the semantics actions are used to construct the AST for the program.
The abstract syntax was provided with the handout file \texttt{absyn.sml}. This file provided the types of the different nodes that can be used when creating an AST from a tiger program. 

\begin{lstlisting}[frame=single]
Listing test
\end{lstlisting}

\section{Conflict management}
%How conflicts where detected and what you did to resolve them. If you were unable to remove all conflicts, state why they are benign.

\subsection{lvalue}
lvalue was the primary example of association and gramma not always matching, as the following code:
\begin{lstlisting}[frame=single]
lvalue : ID (...)
	   | subscript (...)
	   | fieldExp (...)

subscript : lvalue LBRACK exp RBRACK (...)

fieldExp : lvalvue DOT ID (...)

\end{lstlisting}

This however produces a shift/reduce error, so we will need to fix it by making the lvalue nonterminal right associative. And we will do this by introducing a tail.

\begin{lstlisting}[frame=single]
lvalue : ID ltail (...)

ltail : subscript ltail (subscript::ltail)
      | fieldexp ltail (fieldexp::ltail)
      | ([])
\end{lstlisting}

this will work, as we're merely now able to create the same lvalue, just from another direction, now going from left to right, rather than from right to left.
As an example, I will make the string ID, sub, sub, field, sub. first using the first implementation of lvalue, and the second time with the updated version with a tail.

\begin{lstlisting}[frame=single]
lvalue -> subscript  = lvalue2 subscript
lvalue2 -> fieldExp  = lvalue3 fieldExp subscript
lvalue3 -> subscript = lvalue4 subscript fieldExp subscript
lvalue4 -> subscript = lvalue5 subscript subscript fieldExp subscript
lvalue5 -> ID        = ID subscript subscript fieldExp subscript
\end{lstlisting}
And now with the updates Version
\begin{lstlisting}[frame=single]
lvalue -> ID tail1
tail1  -> subscript = ID subscript tail2
tail2  -> subscript = ID subscript subscript tail3
tail3  -> fieldExp  = ID subscript subscript fieldExp tail4
tail4  -> subscript = ID subscript subscript fieldExp subscript tail5
tail5  -> NONE      = ID subscript subscript fieldExp subscript
\end{lstlisting}
As seen, the two final strings are the same, and doesn't provide a shift/reduce conflict.

\subsection{Mutual recursive function declarations}
Mutual recursive types are implented as they're explained in the apendix, which also proved to be quite the task, as they function in quite a specific way.
Functions, Variables and Type are mutually recursive as long as they're declared by a consecutive sequence of the same kind. 

To implement this we did the following:
1. Identify the kind of element that's found first in the let expression.
2. Make a list of the consecutive elements of the same kind.
3. pattern match to check whether a new kind of element is found next, if so, begin a new list with this kind.
4. Repeat 2 and 3 until no more elements left.

to implent this in our parser, we did the following code:
\begin{lstlisting}[frame=single]
letexp : LET deccon IN seqsexps END (...) )

deccon : fundeclbegin (...)
        | vardeclbegin (...)
        | tydeclbegin (...)

fundeclbegin : fundeclist ([fundeclist])
           | fundeclist tydeclbegin (fundeclist::tydeclbegin)
           | fundeclist vardeclbegin (fundeclist::vardeclbegin)

vardeclbegin : vardec ([vardec])
           | vardec fundeclbegin (vardec::fundeclbegin)
           | vardec tydeclbegin (vardec::tydeclbegin)
           | vardec vardeclbegin(vardec::vardeclbegin)

tydeclbegin : tydeclist (tydeclist)
          | tydeclist vardeclbegin (tydeclist::vardeclbegin )
          | tydeclist fundeclbegin (tydeclist::fundeclbegin )

fundeclist : fundec ([fundec])
           | fundec fundeclist (fundec :: fundeclist)

tydeclist : tydec ([tydec])
          | tydec tydeclist (tydec :: tydeclist)

\end{lstlisting}
Note however that vardec isn't made into a list, this is because value doesn't follow the same rules as Function and Types, as also seen in the absyn.sml file. As vardecldata isn't a list.

\begin{lstlisting}[frame=single]
and decl     = FunctionDec of fundecldata list
             | VarDec of vardecldata
             | TypeDec of tydecldata list
\end{lstlisting}

\subsection{If-Else conflict}

\section{Problems encountered \& experience gained}
%If you had any problems in the work process, write it here with what you gained.
\subsection{Solving conflicts}

\subsection{Creating the abstract syntax}


\section{5 tiger programs}
%5 interesting tiger programs that you find test your parser in a good way.

%You should also create 5 small Tiger programs containing syntax errors, and describe how your parser reacts to them in your report. Make sure that your implementation works on the provided test files

These five tiger programs presented here are created simple in order to focus on different aspects of the implemented parser. These tests contain different syntax errors as required by the project description. 

\subsection{Test1}

\subsection{Test1}
\subsection{Test1}
\subsection{Test1}
\subsection{Test1}


\section{Conlusion}

\end{document}