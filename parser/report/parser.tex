\documentclass{article}
\usepackage[margin=1in]{geometry}
\usepackage{listings}
\title{dOvs - Parsing}

\author{
  Group 9 \\
  Miran Hasanagi\'{c} - 20084902 \\
  Jakob Graugaard Laursen - 20093220\\
  Steven Astrup S\o rensen - 201206081
}

\begin{document}
\maketitle

\section{Introduction}
This report describes our approach to develop the parser, which is the first component in the compilation process. This parser is implemented using the ML-Yacc, as it is suggested by the book and the project description.

The first three section describe different parts of the parser development.
First, important parts of the tiger grammar are introduced with respect to the ML-Yacc. Secondly, it is described how the semantics actions are used to construct the AST for the program. Then it is shown how conflicts where detected, and what was done in order to resolve them.

Following these sections an overview of the problems encountered and experience gain during the work process is provided. Finally, five tiger programs are provided as test cases for different aspects of the developed parser.

\section{The tiger grammar}
%You should not write how you extracted the grammar from the appendix.

\begin{lstlisting}[frame=single]
Listing test
\end{lstlisting}

\section{The abstract syntax}
%Write how the semantics actions are used to construct the AST for the program.

\section{Conflict management}
%How conflicts where detected and what you did to resolve them. If you were unable to remove all conflicts, state why they are benign.

\section{Problems encountered \& experience gained}
%If you had any problems in the work process, write it here with what you gained.

\section{5 tiger programs}
%5 interesting tiger programs that you find test your parser in a good way.

\end{document}